\begin{abstract}

Parameter identification is crucial for ensuring the permanence and stability of offshore structures. However, the uncertainties associated with parameters in the field significantly impact the behaviors of the piles. Uncertainties in the geotechnical context arise from several factors, such as differences between in-situ tests and laboratory experiments, spatial variations in the soil profile, and the rationality of the constitutive model. Additionally, uncertainties are also linked to changes in dimensions during the construction phase and variations in the magnitude of design loads. Manual back analysis or traditional design method based on Monte Carlo, are time-consuming and labor-intensive. Although the Bayesian framework offers insights into understanding and updating uncertainties, the computational demands of inference analysis pose substantial challenges for predictive purposes.

Nowadays, data-driven methodologies have permeated various engineering domains, facilitating automatic evolution over time to mitigate uncertainties. However, in the realm of geotechnical engineering, there exists a notable absence of guidelines for data-driven uncertainty quantification (UQ). Starting from a pile issue, this thesis endeavors to address this gap by delving into the latest research and hopes to put forth a comprehensive UQ design framework that can be generally applied in geotechnical engineering. 



\end{abstract}