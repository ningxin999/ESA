\addcontentsline{toc}{chapter}{Abstract}

\begin{abstract}

Proper estimation of offshore piles is vital to the life and permanence of the foundation in question. However, due to the uncertainties of soil parameters in the field, the offshore piles are greatly affected. The source of soil uncertainties may come from various reasons, such as  lack of uniformity between in-situ test and laboratory  experiment; spatial variability of soil profile and rationality of the constitutive model, etc. Traditional statistical analysis which is based on Monte Carlo,  is time-consuming and laborious. Though Bayesian theorem provides ways to understand and update the uncertainties, the amount of the inference analysis is still computationally heavy, thus bringing big challenges for the pile design. Additionally, to mimic the geotechnical structures and bearing behaviors as accurate as possible, digital twin is seeping into all kinds of engineering problems, enabling to evolve over time to persistently represent a unique physical asset and achieving data-driven decision making process. However, state-of-the-art digital twins are still relying on considerable expertise and deployment resources, leading to an only one-off implementation and remaining limitation on providing adaptive digital models on unique offshore piles.


In this thesis, we hope to introduce probabilistic graphical model (PGM) involving in Bayesian inverse analysis to speed up the calculation and provide reasonable posterior results for the soil parameters. Besides, as a mathematical and rigorous foundation, partially observed PGM is proposed to support the transition from custom defined model towards accessible digital twins at scale. Based on such flexible asset-specific models, the entire loading life-cycle can be incorporated into a digital twin forming a unified and accessible foundation for a wide range of offshore piles. Combined with monitored data, the proposed dynamic updated digital twin provides rapid analysis results for reliable soil parameters and enables intelligent decision making on the pile bearing behaviors. 


\end{abstract}