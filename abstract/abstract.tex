\begin{abstract}

The identification of representative parameters is an essential component in performing numerical modelling of geostructures. Uncertainties in the geotechnical context arise from several factors, such as differences between in-situ tests and laboratory conditions, spatial variations in the soil profile, and the choices of the constitutive model. Numerical modelling of geostructures is further complicated by uncertainties resulting from the choice of stages construction or complex physics. Traditional methods of manual back analysis or probabilistic design methods based on Monte Carlo simulation to understand the inherent variability in the ground are either time-consuming or computationally expensive. A Bayesian probabilistic framework offers an alternative approach to both manual and Monte Carlo type back-analysis for ground characterisation. The approach offers a mathematically robust framework for incorporating prior knowledge and data assimilation.

Nowadays, data-driven methodologies have permeated various engineering domains. However, in the realm of geotechnical engineering, there exists a notable absence of guidelines for data-driven uncertainty quantification (UQ). This thesis endeavors to address this gap by delving into the latest research and hopes to put forth a comprehensive UQ design framework that can be generally applied in geotechnical engineering. The efficacy of the proposed framework will be assessed by comparing the data-driven inferred soil parameters against those obtained by manual back-analysis. As a first step, this thesis will present the results and findings relevant to the PISA project, a combined field testing and numerical modelling study into the design of piles subjected to lateral loading, with the aim of developing a comprehensive data set for applying the data-driven uncertainty quantification approach.



\end{abstract}