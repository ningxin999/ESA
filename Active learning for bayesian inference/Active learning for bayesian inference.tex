\chapter{Active learning for sequential Bayesian inference}


\section{Sequential enrichment for surrogate model}

Traditional large-scale physics-based models are intractable to solve real-time, many-query context problem. 


Instead of sampling the whole experimental design at once, it has been proposed to use sequential enrichment. Starting with
a small experimental design, additional points are chosen based on the last computed sparse
solution. In the context of machine learning, sequential sampling is also known as active learning.  In all cases, numerical examples show that the sequential strategy generally leads to solutions with
a smaller validation error compared to non-sequential strategies

